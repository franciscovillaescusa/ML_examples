\documentclass[12pt,a4paper]{paper}
\pdfoutput=1
%\usepackage{jcappub}
%\usepackage{times}

%\usepackage{epstopdf}
\usepackage{graphicx}% Include figure files
\usepackage{dcolumn}% Align table columns on decimal point
\usepackage{bm}% bold math
\usepackage{color}
\usepackage{hyperref}
\usepackage{multirow}
\usepackage{empheq}
\usepackage{amsmath}
\usepackage{xcolor}
\usepackage{bm}
%\usepackage{unicode-math}
\input epsf




\newcommand{\apj}[0]{Astrophysical Journal}
\newcommand{\apjl}[0]{Astrophysical Journal Letters}
\newcommand{\aap}[0]{Astronomy and Astrophysics}
\newcommand{\aaps}[0]{Astronomy and Astrophysics Supplement Series}
\newcommand{\apjs}[0]{Astrophysical Journal Supplement}
\newcommand{\aj}[0]{The Astronomical Journal}
\newcommand{\araa}[0]{Annual Review of Astronomy \& Astrophysics}
\newcommand{\AeA}[0]{Astronomy \& Astrophysics}
\newcommand{\mnras}[0]{Montly Notices of the Royal Astronomical Society}
\newcommand{\AeAS}[0]{A\&AS}
\newcommand{\AAS}[0]{AAS}
\newcommand{\ASP}[0]{ASP}
\newcommand{\ASPp}[0]{ASPp}
\newcommand{\AdSpR}[0]{AdSpR}
\newcommand{\SPIE}[0]{SPIE}
\newcommand{\AsNa}[0]{Astronomische Nachrichten}
\newcommand{\pasp}[0]{Publications of the Astronomical Society of the Pacific}
\newcommand{\fcp}[0]{Fundamentals of Cosmic Physics}
\newcommand{\nat}[0]{Nature}
\newcommand{\apss}[0]{Astrophysics and Space Science}
\newcommand{\arxiv}[0]{ArXiv e-prints}
\newcommand{\actaa}[0]{Acta Astronomica}
\newcommand{\ssr}[0]{Space Science Reviews}
\newcommand{\physrep}[0]{Physics Reports}
\newcommand{\prd}[0]{Physical Review D}
\newcommand{\prc}[0]{Physical Review C}
\newcommand{\pasj}[0]{Publ. of the Astronomical Society of Japan}
\newcommand{\na}[0]{New Astronomy}
\newcommand{\jcap}[0]{Journal of Cosmology and Astroparticle Physics}


\newcommand \Mnu{$\Sigma_i m_{\nu_i}$ }
\newcommand \be{\begin{equation}}
\newcommand \ee{\end{equation}}


\begin{document}

%\title{Relation between the 1D and 3D P(k)}
%\author{Francisco Villaescusa-Navarro,}  

  
%\maketitle

%In these notes we derive the relation between the 1D and 3D power spectra and discuss the correct way to compare the 1D power spectrum measured from N-body simulations to the one derived from the 3D one.



If you have an 2D or 3D field with ${\rm SIZE_{IN}}\times{\rm SIZE_{IN}}$, and apply CNNs, the 

\begin{equation}
{\rm SIZE_{OUT}} = \frac{{\rm SIZE_{IN}} + 2{\rm PADDING} - {\rm KERNEL}}{{\rm STRIDES}} + 1
\end{equation}

\vspace{1cm}

For Transposed convolutional neural nets the expression reads
\begin{equation}
{\rm SIZE_{OUT}} = ({\rm SIZE_{IN}-1}){\rm STRIDES} - 2{\rm PADDING} + {\rm KERNEL}
\end{equation}

%\appendix


%\bibliographystyle{JHEPb}
\bibliographystyle{unsrt}
\bibliography{references} 






\end{document}
